\section{User characteristics}

The main actors who interact with the application will be presented under this section. 

\begin{itemize}
    \item \emph{Visitors}: They are not registered users, the system allows them to register through a sign up service either to Data4Help, becoming Individuals or Third-parties, or to Track4Run, becoming organisers. The application lets them also to be spectators of a run organised with Track4Run service. 
    \item \emph{Individuals}: They are single users registered to Data4Help service. It means that the system lets them sign in, accept or refuse Third-parties requests and monitor data acquired from them.
    \item \emph{Third-parties}: They are companies that are registered to Data4Help. Also in this case, the application provides a sign in service, so as to let companies interact with the application and perform single and groups requests. 
    \item \emph{Ambulance service}: The software has to interact with the ambulance service  to provide a non-‐intrusive SOS service to Elderly people, in case of illness. The application sends the exact position where the ambulance has to go.  
    \item \emph{Elderly people}: They are Individuals, the software allows them to subscribe to an SOS service provided by AutomatedSOS, through their Data4Help account (they must have one). They are assumed to be over 65 years old, along with the date of the Elderly person in need of help.
    \item \emph{Map service}: The application needs to interact with the map service, \emph{Google Maps}, in order to let Visitors and other users see on a map the position of all runners during the run.
    \item \emph{Organisers}: They are companies that are registered to Track4Run service. The system provides them with a sign in interface and, after log in, lets them organise runs. They don't need to be registered to Data4Help.
    \item \emph{Runners or Participants}: They are Individuals, the application allows them to enrol in a race through their Data4Help account (they must have one).
\end{itemize}