\section{Assumptions, dependencies and constraints}
    In this section all the domain assumptions made in order for the software-to-be to work are listed, together with the dependecies and the constraints it has to respect.
    
    \subsection{Domain assumptions}
        It is assumed that the following properties hold in the world.
        
        \subsubsection{Data4Help}
        \begin{enumerate}[label={[}D1.\arabic*{]}]
            \item Acquired data is precise enough.
            
            \item Each Individual can be identified unambiguously through his/her fiscal code.
            
            \item One anonymized, data cannot be associated to specific Individuals.
            
            \item Every interaction gets correctly encoded.
            
            \item Third-partied group queries are accepted only if the number of results is greater than 1000.
            
            \item Third-parties are assumed to be companies that want to gather data, therefore they have a VAT code.
            
            \item Third-parties know the fiscal code of the Individual they are looking for.
        \end{enumerate}
        
        \subsubsection{AutomatedSOS}
            \begin{enumerate}[label={[}D2.\arabic*{]}]
                \item An ambulance service capable of handling requests exists. It is also capable of receiving the personal data of the Elderly person and his location.
                
                \item The thresholds beyond which a person is considered to be in need of help are under 40 bpm and above 130 bpm.
                
                \item Elderly people are over 65 years old.
                
                \item The service is offered in Italy (118 is the emergency number for the ambulance service).
            \end{enumerate}
            
            \subsubsection{Track4Run}
                \begin{enumerate}[label={[}D3.\arabic*{]}]
                    \item Maps and tracking services are accurate enough.
                    
                    \item The defined path exists.
                \end{enumerate}
                
    \subsection{Dependencies}
        It has to be noted that both \emph{AutomatedSOS} and \emph{Track4Run} need \emph{Data4Help} in order to work. In particular, from the point of view of the two applications, these assumptions have to hold:
        \begin{enumerate}[label={[}D2-3.\arabic*{]}, leftmargin=*]
            \item Customers (only Participants for \emph{Track4Run}) have to be registered to \emph{Data4Help}.
            
            \item User data is gathered through \emph{Data4Help}.
            
            \item Data gathered through \emph{Data4Help} is valid. \\
        \end{enumerate}
        Moreover, the services refer to Google Maps for the provision of maps.
        
    \subsection{Constraints}
        A smart-watch or a fitness-band are needed in order to use the software and the health data provided is limited to what can be gathered from these devices and to what can be directly calculated from their measuring.
