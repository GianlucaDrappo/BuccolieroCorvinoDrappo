\section{Document Structure}
    The structure of this RASD documents follows IEEE standard for the most part, therefore it is divided into six chapters:
    \begin{enumerate}
        \item Introduction
        
        \item Overall description
        
        \item Specific Requirements
        
        \item Formal Analysis using Alloy
        
        \item Effort Spent
        
        \item References
    \end{enumerate}
    \\
    The overall aim of this document is to state what are the requirements of the software \emph{TrackMe} wants to develop.
    
    In the first chapter a general overview of the purpose of the software-to-be has been provided, together with the description of the context in which it is going to operate. Being an introductory chapter, all the terminology that will be used in the rest of the document has been clearly defined.
    
    In the second chapter a broader description of the software is provided. Here, a class diagram that specifies the relations of the software-to-be with the world can be found, together with the detailed definition of the the assumptions made, the users of the software and its functions.
    
    The third chapter delves into the details of the specification, talking about the various interfaces offered by the software-to-be, defining both its functional requirements through use cases, sequence and activity diagrams, and its non-functional requirements, along with the constraints it has to respect. In this part, differently from the IEEE standard, the description about Performance Requirements is included with the other System Attributes, in order to make the document more homogeneous.
    
    The fourth chapter presents the formal analysis of the most critical parts of the software-to-be using an Alloy model, also showing a world obtained by running such model.
    
    The fifth chapter contains the effort spent by each member of the group in order to realise the RASD.
    
    In the sixth and last chapter all the sources of information exploited to write the present document are listed.