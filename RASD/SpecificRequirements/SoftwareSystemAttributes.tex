\section{Software System Attributes}
    \subsection{Performance requirements}
        The non functional requirements of each service are listed below.
        \subsubsection{Data4Help}
            \begin{itemize}
                \item The frequency of data acquisition must be of one every 30 seconds.
                \item Each request and response must be handled in less than 3 seconds.
            \end{itemize}
            
        \subsubsection{AutomatedSOS}
            \begin{itemize}
                \item The reaction time of the service, from the moment the parameters are measured to be beyond the threshold to when the ambulance is called, must be of less than 5 seconds.
            \end{itemize}
            
        \subsubsection{Track4Run}
            \begin{itemize}
                \item The tracking of the Participants must happen in real time.
            \end{itemize}
            
    \subsection{Reliability}
        The software must be robust and fault tolerant: it is important that the system does not lose information if some component fails. This may require data redundancy.
        
    \subsection{Availability}
        The software must be available 24/7, a part from when it is under maintenance. These interventions have to be scheduled, when possible, and are expected to be performed once per month (e.g.: system upgrade, integration...). 
        
    \subsection{Security}
        The software will handle private and sensitive data, therefore the level of security must be adequate. Information must be transmitted through HTTPS, passwords must be hashed and salted, databases must be encrypted with the latest standards.
    
    \subsection{Maintainability}
        The software should be organised in modules and be well documented, in order to make maintenance, upgrades and integration of new features easy.
    
    \subsection{Portability}
        The software must be able to run under different versions of Android and iOS environments, as already stated in section \ref{OS}.
            