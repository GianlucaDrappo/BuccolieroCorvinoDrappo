\section{Document Structure}
    The structure of this DD document follows IEEE standard, therefore it is divided into six chapters:
    \begin{enumerate}
        \item Introduction
        
        \item Architectural Design
        
        \item User Interface Design
        
        \item Requirement Traceability
        
        \item Implementation, Integration and Test Plan
        
        \item Effort Spent
    \end{enumerate}
    
The overall aim of this document is to describe the architecture and design of the applications \emph{TrackMe} wants to develop.
    
    In the first chapter a general overview of the purpose of the software-to-be has been provided, together with the description of the context in which it is going to operate. Being an introductory chapter, all the terminology that will be used in the rest of the document has been clearly defined.
    
    In the second chapter the architectural design of the software-to-be is described, highlighting the components it is going to be made of, the way they will be deployed, the interaction among them and the interfaces through which they will communicate. Furthermore, the architectural styles and patterns are listed and justified here.
    
    In the third chapter it is explained how the requirements defined in the RASD map to the design elements defined in the second chapter.
    
    The fourth chapter presents a way in which the implementation of components of the system can be led, along with the integration order of said components. Moreover, a test plan for the application is provided.
    
    The fifth chapter contains the effort spent by each member of the group in order to realise the DD.
    
    The Bibliography is provided after the fifth chapter. All the sources of information exploited are listed there.
    
    The Appendix \ref{GoalRequirementsDomainAssumptions} in the end of the document contains the goals, the requirements and the domain assumptions as they have been formalised in the RASD.
    
    For a description about the User Interface, refer to the Requirements Analysis and Specification document, section 3.1.1.
    
    