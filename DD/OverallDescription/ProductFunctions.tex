\section{Product functions}

In this section there is an analysis of functional requirements (grouped by the three services) that the application needs to reach the main goals. 



\subsection{Data4Help}
\begin{enumerate}[label={[}R1.\arabic*{]}, leftmargin=*]

    \item \label{R1-Individual-registration} The system allows Individuals to register, requiring name, surname, fiscal code, password and email in order to create a new Individual account.

    \item \label{R1-third-party-registration} The system allows Third-parties to register, requiring the company name, VAT number, email and password in order to create a new Third-party account.
    
    \item \label{R1-gathered-data} The application retrieves from smart-watches the following data, associating them to the user: 
    \begin{itemize}
        \item beats per minute ( bpm );
        \item number of steps;
        \item location;
        \item time-stamp of the measurement.
    \end{itemize}
    
    \item \label{R1-individual-request} The system allows Third-parties to request data of Individuals and the latter to accept or refuse them. 
    \begin{enumerate}[label={[}R1.\arabic{enumi}.\arabic*{]}, leftmargin=*]
        \item \label{R1-individual-accept-request} If the Individual accepts, the system sends to the Third-party all the data gathered from the specified user.
    
        \item \label{R1-individual-refused-request} If the Individual refuses or he/she is not registered, the system sends to the Third-party an error message.
    \end{enumerate}
    
    \item \label{R1-group-requests} Upon Third-parties request, the system can perform parametric searches based on geographical areas, age, genre, time of the day. The result of the query is then stored and evaluated for approval.
    \begin{enumerate}[label={[}R1.\arabic{enumi}.\arabic*{]}, leftmargin=*]
        \item \label{R1-group-req-approuved} If the request is approved, the saved data is provided to the Third-party.
        \item \label{R1-group-request-refused} If the request is not approved, an error message is sent to the Third-party.
        \item \label{R1-group-data-anonymized} In order to anonimize data, the application only sends information related to the health status of individuals (e.g. bpm, step, genre, age).
    \end{enumerate}
        
    \item \label{R1-third-party-subscription} The application allows Third-parties to subscribe to certain data.
    \begin{enumerate}[label={[}R1.\arabic{enumi}.\arabic*{]}, leftmargin=*]
        \item \label{R1-subscription-updates} The application, if the Third-party subscribed to some data (belonging to Individuals or group searches), sends updates to the Third-party aggregating the data with the specified granularity and frequency.
    \end{enumerate}
        
    \item \label{R1-unsubscription} The application allows Individuals withdrawals of consent for accessing their data and Third-party unsubscriptions.
    \begin{enumerate}[label={[}R1.\arabic{enumi}.\arabic*{]}, leftmargin=*]
        \item \label{R1-Individual-unsubscription} If the Individual withdraws consent for the access of his data, a message is sent to the subscribed Third-party and data is no longer provided.
        \item \label{R1-third-party-unsubscription} If the Third-party no longer wishes to collect data from an Individual, a message is sent to him/her and his/her data is no longer sent.
    \end{enumerate}
        
        
    \item \label{R1-history-request} The system allows Third-parties to access to their request history and results.
    
    \item \label{R1-individual-personal-data} The system allows Individuals to check gathered data.

\end{enumerate}


\subsection{AutomatedSOS}

\begin{enumerate}[label={[}R2.\arabic*{]}, leftmargin=*]

    \item \label{R2-data-and-ambulance} The system has to analyse user data and sends the location of the user to the ambulance service in case of emergency, when health values goes beyond the threshold.
    
    \item \label{R2-subscription} The system gives the possibility to subscribe to the service upon request, only after checking the Individual age from his/her fiscal code.
\end{enumerate}


\subsection{Track4Run}

\begin{enumerate}[label={[}R3.\arabic*{]}, leftmargin=*]

    \item \label{R3-login-enrollin} The system allows Participants to log into the application through their Data4Help account and, after they are logged in, to enrol the the runs available. They should also be able to apply filters to ease the research (e.g.: place, time, date range).
    
    \item \label{R3-enrolling-different-time} The system allows the enrolling of Participants only if they are not enrolled in a race at the same time.
    
    \item \label{R3-visitors} The system allows Visitors to access the application as host-users, without registration, by giving them only the possibility to track the Participants.
    
    \item \label{R3-ranking-end} At the end of each run, the systems shows the rank to all Users watching it.
    
    \item \label{R3-ranking-storage} The system stores the ranks of finished races so that both Participants and Organisers can access to the race history.
    
    \item \label{R3-organizer-registration} The system allows the registration of Organisers without requiring also Data4Help registration.
    
    \item \label{R3-run-organization} In order to organise a run, the application requires location, data, time and the maximum number of participants.
    
    \item \label{R3-decreasing-participant-number} The system needs to update subscriptions to the runs, decreasing the number of allowed participants. It must not allow further subscriptions if there are no leftover places.
    
    \item \label{R3-run-different-time} The system does not allow that two or more runs are organised at the same place and time.

\end{enumerate}